\documentclass[a4paper,12pt]{article}
%\usepackage[brazilian]{babel}
\usepackage[table]{xcolor}
\usepackage[utf8]{inputenc}
\usepackage{amsmath,amsthm,amsfonts,amssymb,amscd,graphicx} 
\usepackage{indentfirst}
\usepackage{float}
\usepackage{fancyhdr}
\usepackage[inline]{enumitem}
\usepackage{multicol}
\usepackage{hyperref}
\usepackage{tcolorbox}
\usepackage[top=1.4cm,left=1.5cm,right=1.5cm,bottom=2.5cm]{geometry}
\usepackage{parskip}
\usepackage{tikz}

\newtcolorbox{mybox1}[1]{colback=red!5!white,colframe=red!75!black,fonttitle=\bfseries,title=#1}
\newtcolorbox{mybox2}[1]{colback=blue!5!white,colframe=blue!75!black,fonttitle=\bfseries,title=#1}

\newtheorem{theorem}{Teorema}[section]
\newtheorem*{example}{Exemplo}
\newtheorem*{examples}{Exemplos}


\newcommand{\HRule}{\rule{\linewidth}{0.5mm}}
\setlength{\parskip}{1em}

\begin{document}
	
	\begin{center}
		{\large \bf Universidade Federal de Lavras \\ GEX102 - Geometria Analítica e Álgebra Linear}
		\vskip-.5cm
		\noindent $\hrulefill$
	\end{center}
	\vskip.5cm
	
	{\Large \bf Orientações 1 - Matrizes e Operações. Sistemas Lineares.}
	\vskip.5cm
	
	Estas orientações são para as duas primeiras semanas do curso. 
	Na primeira semana, trabalharemos com o conteúdo de matrizes e operações.
	Os seus objetivos são:
	\begin{mybox2}{}
	\begin{itemize}
		\item aprender o que é uma matriz e entender as notações envolvendo elementos de matrizes. 
		\item saber o nome de diferentes tipos de matrizes
		(matriz nula, quadrada, triangular superior, inferior, diagonal);
		\item ser capaz de construir matrizes através da definição de suas entradas; 
		\item aprender a fazer as seguintes operações com matrizes (somar, subtrair, multiplicar,
		multiplicar por escalar e transpor).
		\item estudar as propriedades das operações; (estas irão aparecer durante as próximas sema\-nas.)
		\vskip.5cm
%		Em resumo, conhecer o que são matrizes e aprender a efetuar manualmente as operações
%		entre elas.
	\end{itemize} 
	\end{mybox2}

	Na semana seguinte, o conteúdo abordado será a relação de matrizes com sistemas lineares.
	Aqui, seus objetivos serão:
	
	\begin{mybox2}{}
	\begin{itemize}
		\item saber o que é um sistema de equações lineares e como reescrevê-lo usando matrizes;
		\item aprender a resolver um sistema de equações lineares na forma matricial usando
		o método de escalanamento (método de Gauss) e o método de escalonamento reduzido (método
		de Gauss-Jordan);
	\end{itemize}
	\end{mybox2}
	O conteúdo você encontra no livro-texto{\footnote{SANTOS, Reginaldo J. Matrizes, Vetores e Geometria Analítica. Belo Horizonte,	MG: Imprensa Universitária da UFMG, 2007.
	Acesse o pdf \href{https://www.dropbox.com/s/aa71ogpk8xski1j/gaalt1.pdf?m}{\bf \color{blue}aqui.}}}
	do curso, nas seções 1.1, 1.1.1, 1.1.2, 1.2.1.
	
	Além disso, no campus virtual você encontrará o arquivo pdf do livro-texto 
	(ou \href{https://www.dropbox.com/s/aa71ogpk8xski1j/gaalt1.pdf?m}{\bf \color{blue}clicando aqui}),
	notas de aulas/slides dos
	professores, vídeos indicados, vídeos gravados pelos professores do curso e listas de exercícios com gabarito. Além disso toda semana,
	os professores e monitores estarão disponíveis para explicação do conteúdo e dúvidas 
	(Consultar horários no Campus Virtual).
	
	Faremos uma transmissão ao vivo, na terça-feira (27/10) às 10horas pelo youtube (Link: 
	\href{https://www.youtube.com/watch?v=w3ft57gfF60}{Live 1} ).
	
	\begin{mybox1}{{\bf Avaliações do conteúdo:}}
	 Serão aplicados 2 questionários online: 
	\begin{itemize}
		\item revisão 1: questionário de revisão sobre matrizes e sistemas lineares; 
		2 pontos; Infinitas chances;
		\item Teste 1: avaliação do conteúdo das duas primeiras semanas; 8 pontos; 1 chance;
	\end{itemize}
	$^*$As datas estarão na descrição das atividades no Campus Virtual. Não necessariamente o período de início ou término das atividades corresponderá ao das semanas do conteúdo.

	\end{mybox1}

	
\end{document}